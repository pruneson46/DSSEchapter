%%%%%%%%%%%%%%%%%%%% author.tex %%%%%%%%%%%%%%%%%%%%%%%%%%%%%%%%%%%
%
% sample root file for your "contribution" to a contributed volume
%
% Use this file as a template for your own input.
%
%%%%%%%%%%%%%%%% Springer %%%%%%%%%%%%%%%%%%%%%%%%%%%%%%%%%%


% RECOMMENDED %%%%%%%%%%%%%%%%%%%%%%%%%%%%%%%%%%%%%%%%%%%%%%%%%%%
\documentclass[graybox]{svmult}

% choose options for [] as required from the list
% in the Reference Guide

\usepackage{type1cm}        % activate if the above 3 fonts are
                            % not available on your system
%
\usepackage{makeidx}         % allows index generation
\usepackage{graphicx}        % standard LaTeX graphics tool
                             % when including figure files
\usepackage{multicol}        % used for the two-column index
\usepackage[bottom]{footmisc}% places footnotes at page bottom


\usepackage{newtxtext}       % 
\usepackage{newtxmath}       % selects Times Roman as basic font

% see the list of further useful packages
% in the Reference Guide

\makeindex             % used for the subject index
                       % please use the style svind.ist with
                       % your makeindex program

%%%%%%%%%%%%%%%%%%%%%%%%%%%%%%%%%%%%%%%%%%%%%%%%%%%%%%%%%%%%%%%%%%%%%%%%%%%%%%%%%%%%%%%%%

%(1)    The formatting templates (in both LaTeX and Word) are available under [1]. The current LaTeX template is directly available at [2] (please use the template from the subdirectory “author” in the zip file) and the current Word template at [3].
%(2)    The chapters should be written in a clear, consistent, self-contained textbook-like style. Therefore, please comply with the following structure:
%a.       Each chapter should start with an “Introduction” and end with a short “Conclusion”. Please also consider a section “Recommended Further Reading” before the Conclusion.
%b.       The goal of each chapter is to give a comprehensive overview on a contemporary topic in empirical software engineering (for your convenience you find below a list of the confirmed chapter topics)
%c.       Please provide examples and evidence
%(3)    The intended length of your chapter should be approximately 25 pages.
%(4)    The deadline to submit the first version of the chapter is the 30th of April 2019.



\begin{document}

\title*{Design Science and Continuous Empirical Software Engineering}
% Use \titlerunning{Short Title} for an abbreviated version of
% your contribution title if the original one is too long
\author{Per Runeson, Emelie Engstr\"om and Margaret Anne Storey}
% Use \authorrunning{Short Title} for an abbreviated version of
% your contribution title if the original one is too long
\institute{Per Runeson and Emelie Engst\"om \at Lund University, Sweden, \email{[per.runeson;emelie.engstrom]@cs.lth.se}
\and Margaret Anne Storey \at University of Victoria, Canada \email{mstorey@uvic.ca}}
%
% Use the package "url.sty" to avoid
% problems with special characters
% used in your e-mail or web address
%
\maketitle

\abstract*{Each chapter should be preceded by an abstract (no more than 200 words) that summarizes the content. The abstract will appear \textit{online} at \url{www.SpringerLink.com} and be available with unrestricted access. This allows unregistered users to read the abstract as a teaser for the complete chapter.
Please use the 'starred' version of the \texttt{abstract} command for typesetting the text of the online abstracts (cf. source file of this chapter template \texttt{abstract}) and include them with the source files of your manuscript. Use the plain \texttt{abstract} command if the abstract is also to appear in the printed version of the book.}

\abstract{Each chapter should be preceded by an abstract (no more than 200 words) that summarizes the content.}

\section{Introduction}
\label{sec:intro}

\section{Recommended Further Reading}
\section{Conclusion}

\bibliographystyle{plain}
\bibliography{dsse}
\end{document}
